%**********************************************************************************
%
% o   o   o   o          Berne University of Applied Sciences
%             :          Engineering and Information Technology
%             :......o   Computer Science Division
%
% Airport Simulation 
% Matthias Blaser and Stefan Heinemann
%**********************************************************************************
\documentclass[oneside,DIV12,BCOR0.5cm,bibliography=totoc]{template}
\raggedbottom

\begin{document}
\begin{empfile}
\begin{empcmds}
input metauml;
\end{empcmds}


\title{Aiport Simulation}
\subtitle{P2PMPI-Beispielprojekt}

\author{Matthias Blaser\symbolfootnote[2]{\href{mailto:blasm5@bfh.ch}{blasm5@bfh.ch}} undStefan Heinemann\symbolfootnote[1]{\href{mailto:heins4@bfh.ch}{heins4@bfh.ch}} }

\maketitle


\section{Einleitung}

\section{Messaging}

Falls der Zielflughafen eines Flugzeugs nicht denselben Rank besitzt
wie der Flughafen, der vom aktuellen LP verwaltet wird, wird am
Ende des Start-Vorgangs zusätzlich zum Scheduling des ARRIVAL-Events 
eine Message an den Zielflughafen versendet, welche die wichtigsten
Daten des Flugs enthält. Der kreiert dann bei sich selbst ein 
ARRIVAL-Event. Ab diesem Zeitpunkt ist das Flugzeug beim LP des 
Zielflughafens bekannt und wird, sobald das ARRIVAL-Event
ausgelöst wird, normal abgehandelt. Zudem kann es nun, sobald es
in den sichtbaren Bereich des Flughafens fliegt, gezeichnet werden
vom GUI.\\

Beim Flughafen, von dem das Flugzeug gestartet wird, wird das Flugzeug
beim Auslösen des ARRIVAL-Events aus der Simulation entfernt, da es
ja jetzt nicht mehr von diesem LP behandelt wird.\\

\section{Synchronisierung}

Wir haben die Synchronisierung mit Null-Messages implementiert.

\subsection{Null-Messages}

Gibt es in der Outgoing-Message-Queue keine Messages, die verschickt
werden sollen, verschickt der Send-Thread automatisch in Regelmässigen
Abständen an alle anderen LPs eine Null-Message mit Timestamp und 
Lookahead.\\

Falls eine Message in der Queue ist, wird nebst dieser an die 
verbleibenden LPs ebenfalls eine Null-Message verschickt mit dem
selben Timestamp und Lookahead.

\subsection{Lookahead-Queue}

Beim Erhalt einer Message, egal ob Null-Message oder ''richtig``, 
wird der Timestamp mit dem Lookahead addiert und in der 
\emph{LookaheadQueue} abgelegt. Diese ist dazu da, dass die
Simulation immer abfragen kann, bis zu welchem Zeitpunkt sie sicher
Events verarbeiten darf.

\section{GUI}

\section{Logger}

Um nachvollziehen zu können, was genau in der Simulation auf den
verschiedenen LPs abläuft, haben wir eine Logger-Klasse implementiert,
welche verschiedene Debug-Meldungen in Log-Dateien schreibt. Diese
Log-Dateien werden im Verzeichnis \emph{/tmp} abgelegt und heissen
\emph{airport-<rank>.log}.

\section{MPI-Fix}

Da für die Slick-Library gewisse Native Libraries gebraucht werden,
werden diese ebenfalls beim Start der MPI-Umgebung mit geschickt. Damit
die LPs bei der Ausführung diese Libraries finden, muss beim starten
der Simulation der \emph{java.library.path} an die Java Virtual
Machine mitgegeben werden. Da das Ganze aber über die MPI-Umgebung
gestartet wird, ist dies nicht ohne weiteres möglich wie z.B. bei
der Angabe der Jar-Files, welches einfach über die Umgebungs-Variable
\emph{LD\_LIBRARY\_PATH} gemacht werden kann.\\

Aus diesem Grund haben wir das Script \emph{lib/p2pmpi/bin/p2pclient}
so angepasst, dass die Variable \emph{java.library.path} beim Aufruf
der Java Virtual Machine gesetzt wird. Es ist also erforderlich, dass
die Software mit der mitgelieferten P2P-MPI-Version gestartet wird.


\end{empfile}
\end{document}
