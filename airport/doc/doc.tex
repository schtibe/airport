%**********************************************************************************
%
% o   o   o   o          Berne University of Applied Sciences
%             :          Engineering and Information Technology
%             :......o   Computer Science Division
%
% Airport Simulation 
% Matthias Blaser and Stefan Heinemann
%**********************************************************************************
\documentclass[oneside,DIV12,BCOR0.5cm,bibliography=totoc]{template}
\raggedbottom

\begin{document}
\begin{empfile}
\begin{empcmds}
input metauml;
\end{empcmds}


\title{Aiport Simulation}
\subtitle{P2PMPI-Beispielprojekt}

\author{Matthias Blaser\symbolfootnote[2]{\href{mailto:blasm5@bfh.ch}{blasm5@bfh.ch}} und Stefan Heinemann\symbolfootnote[1]{\href{mailto:heins4@bfh.ch}{heins4@bfh.ch}} }

\maketitle


\section{Einleitung}

Dieses Dokument beschreibt gewisse Aspekte und Überlegungen im Beispielprojekt, 
welches im Zug des Fachs Parallel Computing an der Berner Fachhochschule 
entstanden ist.\\

Der Gebrauch der Software kann in der Datei \emph{README.md} nachgelesen
werden. Bisher wurde sie verteilt auf drei Prozessoren getestet - um
sie mit mehr oder weniger Prozessoren zu starten muss im Sourcecode der
Applikation die Definition der Flughäfen (Namen, Koordinaten auf der
Map usw.) und dessen Zuordnung auf die Prozessoren angepasst werden.


\section{Messaging}

Das Messaging wird von zwei Threads übernommen, die neben der Simulation
laufen. Der eine behandelt das Senden und der andere das Empfangen
von Messages. Dabei gibt es eine Queue für den Send-Thread 
(Outgoing-Message-Queue), wo von der Simulation Nachrichten abgelegt
werden können die verschickt werden sollen.\\

Der Ablauf des Messaging ist wie folgt: Falls der Zielflughafen eines
Flugzeugs nicht denselben Rank besitzt wie der Flughafen, der vom
aktuellen LP\footnote{Logischer Prozessor: der einzelne Prozess, der ein
Teil der Simulation selbständig durchführt} verwaltet wird, wird am Ende
des Start-Vorgangs zusätzlich zum Scheduling des ARRIVAL-Events eine
Message an den LP des Zielflughafen versendet, welche die wichtigsten
Daten des Fluges enthält. Dieser LP kreiert dann bei sich selbst ein
ARRIVAL-Event. Ab diesem Zeitpunkt ist das Flugzeug beim LP des
Zielflughafens bekannt und wird, sobald das ARRIVAL-Event ausgelöst
wird, normal abgehandelt. Zudem kann es nun, sobald es in den sichtbaren
Bereich des Flughafens fliegt, vom GUI gezeichnet werden.\\

Beim Flughafen, von dem das Flugzeug gestartet wurde, wird das Flugzeug
beim Auslösen des ARRIVAL-Events aus der Simulation entfernt, da es
nun nicht mehr von diesem LP behandelt wird.\\

\section{Synchronisierung}

Wir haben die Synchronisierung mit Null-Messages implementiert.

\subsection{Null-Messages}

Gibt es in der Outgoing-Message-Queue keine Messages die verschickt
werden sollen, verschickt der Send-Thread automatisch in regelmässigen
Abständen an alle anderen LPs eine Null-Message mit Timestamp und 
Lookahead.\\

Falls eine Message in der Queue ist, wird nebst dieser ebenfalls eine
Null-Message mit demselben Timestamp und Lookahead wie das entsprechende
``richtige'' Event an die verbleibenden LPs verschickt.

\subsection{Lookahead-Queue}

Beim Erhalt einer Message, egal ob Null-Message oder ''richtig``, wird
der Timestamp mit dem Lookahead addiert und in der \emph{LookaheadQueue}
abgelegt. Diese ist dazu da, dass die Simulation immer abfragen kann,
bis zu welchem Zeitpunkt sie sicher Events verarbeiten darf.\\

Diese ist so implementiert, dass sie für jeden der ''anderen`` LPs
eine Queue bereitstellt, wo die jeweiligen Lookahead-Werte abgelegt
werden. Sie stellt eine Routine bereit, um zu prüfen, ob jede dieser
Queues mindestens einen Lookahead-Wert enthält. Ist dies nicht der Fall,
muss die Simulation angehalten werden, weil die Ausführung von Events
nicht mehr sicher ist.\\

Sind alle Queues mit mindestens einem Wert gefüllt, kann der aktuell
kleinste Lookahead-Wert abgefragt werden (der kleinste Wert aller
kleinsten Werte der Queues). Sind nun Events mit einem kleineren
Timestamp als der kleinste Lookahead in der lokalen Event-Queue
vorhanden, können diese sicher abgearbeitet werden, da sichergestellt
ist, dass jeder andere LP seine lokale Zeit mindestens zu diesem
Timestamp fortgeschritten hat. Ist dies nicht der Fall (keine lokalen
Events vorhanden), wird bis zum Lookahead gewartet und die lokale
Simulationszeit Schritt für Schritt erhöht. Ist ein Lookahead einmal
erreicht (Simulationszeit entspricht dem sicheren Timestamp aus den
Lookahead-Werten), wird er aus den Queues entfernt (und zwar aus allen,
da möglicherweise mehrere LPs den selben Wert verschickt haben).

\section{GUI}

Für die Anzeige der Simulation, also das Zeichnen der Flugzeuge auf der
Karte, wird das 2D-Gaming Framework
Slick\footnote{http://slick.cokeandcode.com/} verwendet. Dieses stellt
alle nötigen Klassen zur Erstellung und Rendering eines Spiels zur
Verfügung und erlaubt eine einfache Ansteuerung des Spielfeldes (Karte).

In der Simulation wird jeweils der Heimatflughafen auf dem Spielfeld
zentriert und die Koordinaten der Objekte (Flughafen und Flugzeuge)
werden beim Platzieren von den Koordinaten der gesamten Karte auf
das Koordinatensystem des Spielfelds umgerechnet. Zudem wird ein 
Status-Text mit der aktuellen Rank-Nummer, der aktuellen
Simulationszeit (CST) und dem kleinsten aktuellen Lookahead-Wert (oder
<none>, falls zur Zeit keiner vorhanden ist). 

\section{Logger}

Um nachvollziehen zu können, was genau in der Simulation auf den
verschiedenen LPs abläuft, haben wir eine Logger-Klasse implementiert,
welche verschiedene Debug-Meldungen in separaten Log-Dateien
schreibt. Diese Log-Dateien werden im Verzeichnis \emph{/tmp} abgelegt
und heissen \emph{airport-<rank>.log}.

\section{MPI-Fix}

Da für die Slick-Library gewisse Native Libraries gebraucht werden,
werden diese ebenfalls beim Start der MPI-Umgebung mitgeschickt. Damit
die LPs bei der Ausführung diese Libraries finden, muss beim Starten
der Simulation der \emph{java.library.path} an die Java Virtual
Machine mitgegeben werden. Da das Ganze aber über die MPI-Umgebung
gestartet wird, ist dies nicht ohne weiteres möglich wie z.B. bei
der Angabe der Jar-Files, welches einfach über die Umgebungs-Variable
\emph{LD\_LIBRARY\_PATH} gemacht werden kann.\\

Aus diesem Grund haben wir die Scripts \emph{lib/p2pmpi/bin/p2pmpirun}
und \emph{p2pclient} so angepasst, dass die Variable
\emph{java.library.path} beim Aufruf der Java Virtual Machine gesetzt
wird. Es ist also erforderlich, dass die Software mit der mitgelieferten
P2P-MPI-Version gestartet wird und, falls die Software auf mehrere
Hosts verteilt wird, die benutzte P2P-MPI-Version ebenfalls gepatcht
ist.

\end{empfile}
\end{document}
